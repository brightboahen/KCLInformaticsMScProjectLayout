\section{Background and Literature Survey}
\subsection{Classroom Seating Arrangements}
Seating arrangements in a classroom are important because they have the potential to help prevent problem behaviours that decrease student attention and diminish available instructional time \cite{wannarka2008seating}. The position in the classroom that a pupil sits and who he or she sits plays an important role in their development and performance. Daniels \cite{daniels1998manage} concluded that the position out of a pupil in a classroom has the potential to produce a desirable behaviour of the pupil or contribute to pupils misbehaviour.

The classroom seating arrangement is under the control of the teacher unlike other factors that contribute to a pupils behaviour or performance in class. There are a lot of evidence to support this, as Moore and Glynn \cite{moore1984variation} established that where a pupil is located in the classroom is related to the number of questions received from the teacher.

We are seeing that it is not only imperative that pupils are placed in the right place in the classroom but a vital decision that must be made by the teacher using their better judgement and theories as found by Wannarka and Ruhl \cite{wannarka2008seating} by arranging seats to suit the nature of task.

\subsection{Student Behaviour on-task}
On-task behaviour is a term used to describe the behaviour of a pupil or student during the performance of a specific task, example students being asked questions and an answer expected of them. Off-task describes when for a example a move away from their table without permission. These definitions have been used in a number of research to find out how best to get the best arrange seating in a classroom so as to improve on-task and decrease off-task behaviours. Studies that have focused on on-task (individual tasks) concluded that seating in rows work best. Marks, Fuhrer and Hartig \cite{marx1999effects} in their studies of a question-asking task of 10 year olds found that they pupils asked their teacher more questions when they were sat in a semi-circle than they did when they were sat in rows.

\subsection{Whole Class Teaching}
There are evidence to support claims that whole class teaching provides order, control, purpose and concentration in classrooms.``Whole class teaching is associated with higher order questioning, explanations and statements and these in turn correlate with higher levels of pupil performance''\cite{alexander1992curriculum}

\subsection{Group Work}
Phil Beadle \cite{website:TES} in his article states how pupils responded to a survey asking them ``How do you learn best?'' with ``We learn best in groups''. Grouping pupil in a classroom encourages shared resources, social development and provides pupil interaction with each other as well as the teacher \cite{alexander1992curriculum}.

OFSTED \cite{OFSTED} in their \emph{Pupil Behaviour in Schools in England} reported that a teacher can get the best out of boys in a mixed school by pairing them with slightly lower ability girls. If they are paired with a higher ability girl they get into a mode whereby they get the girls to do their work for them whereas they get stupid and competitive when paired with girls of the same ability. This theory however only makes sense if the school in question is a mixed school and also for example pupils detesting each other, such a pairing will not work.

\subsection{Existing Systems}
We have seen the importance of classroom seating arrangements and the impact it has on the development and academic performance of pupils. They are a number of systems that operate on this theory and provide classroom management tools for teachers. In this section we will discuss some of these systems and their approach to creating a dynamic and effective classroom.

\subsubsection{ClassCharts} \label{sub:classchart}
ClassCharts \cite{website:ClassCharts} is by far the most complete and feature rich of the classroom management tools on the market at the moment. ClassCharts provides teachers with instant seating plans and behaviour management. They can re-arrange pupils based on settings they can adjust with sliders and they can manually update or rate a pupils behaviour and or performance in the class, this provides a teacher with a report as to who is performing well and in what pairing.

\subsubsection{ClassDojo} \label{sub:classdojo}
ClassDojo \cite{website:classDojo} aims to provide a social platform that brings parents, pupils and teachers together. The system is developed to create a classroom community not just for the pupils but also for their parents. It encourages parents to take part in the classroom. ClassDojo does have a feature that allows the reorganisation or re-arrangement of pupil seating positions but it trys to reinvent the classroom by emphasising on the social element, for example the status of a pupil can be shared on the platform much like a user sharing their status on a social media platform.

\subsubsection{PupilAsset} \label{sub:pupilasset}
This system is a suite of tools that provide tracking abilities in terms of pupil target grade, current grade and progressions. It also provides in depth coverage of results, messaging and attendance among other utility tools \cite{website:PupilAsset}.

\subsubsection{Promethean World} \label{sub:promethean}
Promethean World is one other system that falls into the classroom management tool category; it is a suite of utility products such as interactive whiteboard, interactive tables and student response devices. Their student response devices encourages student classroom participation and real time feedback on student progress \cite{website:promethean}

\subsubsection{Discussion}
In the previous sub sections we looked at some existing systems that aim to encourage effective seating arrangements in the classroom, these systems are very effective and efficient at producing the right content and or result to the user but as stated in the introductory chapter of this paper, these systems are built with a one size fits all ideology. They are designed to be efficient at specified tasks and reduce human errors in the process. \ref{sub:classchart} is a system that creates and organises an optimal seating plan for the user but only after a number of clicks and requiring the user to know exactly what they are doing, that is requiring more cognitive skills per each action without taking into consideration the knowledge the user has of the system after spending some time on the system.

\ref{sub:classdojo} takes a different approach to classroom management, an intuitive approach but it's emphasis on creating a social community in part goes against the research and theories that have been developed over the years to improve pupil performance in the classroom. In an era where social media is at the centre of lives, it would be beneficial if a classroom setting is not a reflection of popular culture.This system also requires more instead of less cognitive skills on the user(teachers) part.

The student response devices provided by \ref{sub:promethean} is one of the revolutionary tools to be introduced in a classroom, but although it encourages student participation in the classroom, it can also discourage pupil to pupil interaction during lessons. Example pupils with an introvert personality can take part in class without feeling exposed, but this would not improve on the pupils social development as founded by \cite{alexander1992curriculum}.

These systems in isolation have their weaknesses and strengths as pointed out but there is a recurrent theme, a constant in all the system and that is they are not designed to adapt to the user, this is mainly due to the costs involved in developing a general purpose system to suit different personalities and stereotypes.

\subsection{A brief history of user interfaces} \label{sub:history}
The concept of \emph{user interfaces} surfaced in the 1960's, when in 1963 Ivan Sutherland published his MIT PhD thesis about a system called \textbf{Sketchpad} \cite{sutherland1964sketch}.

The result of Ivan Sutherland's Sketchpad was a pioneering graphical user interface (GUI). Sketchpad allowed the user to create graphic images directly on the computer screen using a lightpen \cite{patrick2003intelligent}.
Prior to the early 1970's, researchers focused primarily on new technologies but this trend changed in 1970 when the focuse shifted from developing and discovering new technologies to the user's interaction with a machine.
By the 1980's the study of human and computer interaction had changed into a user-centered research field with usability as its main goal and technology as a supporting tool.\cite{patrick2003intelligent}

This led to the emergence of Intelligent User Interfaces(IUI) as a subfield of study of Human-Computer Interaction. IUI came into prominence in the early 1990's with microsoft releasing their office assistant help system in 1997.

In its salient form, an adaptive user interface or system monitors the users interaction with the system or interface and then tries to identify a pattern in the system's usage attributed to user difference s in order to automatically adjust the interface components or content to allow for such user differences as awell as changes in user skills, knowledge and preferences.

We see examples of such mechanisms in our daily usage of our smart phones, laptops and or on our favourite social media platforms. Facebook's advertising feature shows products unique to users based on what their adaptive system has identified as a pattern associated to the user.On our smartphones, features like the predictive text is a form of adaptiveness. It learns and tries to improve a users texting or typing experience.

\subsubsection{Why Adaptive User Interfaces?}
Pioneering computer softwares were designed and developed to solve business and scientific problems in predetermined way that allowed only very constrained user input, through arguments provided to the program at runtime.\cite{langley1997machine}.
If we observe how we interact with computers now and the design of computer software, we can conclude that a lot has changed; software now accept and support frequent user input. Modern day interfaces try to be intuitive by using a desktop metaphor which consists of multiple ``windows'' showing folders and documents \cite{patrick2003intelligent}

On the other hand, one important obstacle in the way of current interactive systems is that they have little ability to take into account differences in the knowledge, style and preferences of the user \cite{langley1997machine}.
Systems like document production (microsoft word) and Enterprise Systems lets a user select a set of predefined or default styles and even store their own variations, but these processes tend to be manual and tedious. \cite{langley1997machine}.

Clearly there is the need for adaptability and personalisation to reduce the manual process and cognitive skills required in using or interacting with computer software. Most applications that have attempted to implement adaptive user interfaces have required the users explicitly state their preferences to the interface. This is a tiresome process and most importantly, some user styles may be reflected in a user's behaviour but not subject to conscious inspection \cite{langley1997machine}.

This has led to the use of Artificial Intelligence techniques to deal with different forms of input and output and to try and help the user in an intelligent way. These interfaces try to solve the problems that current systems cannot. As in \cite{patrick2003intelligent},these include:
\begin{itemize}
\item Creating Personalised Systems; different people form different mental models of an application or system. This needs to be accounted for as what would make complete sense to user A would not make sense to user B.
\item Information overflow ; Information overflow or filtering has been a major problem for direct manipulation systems. This process can be likened to finding a needle in a haystack.
\item IUI provides other forms of interaction, for example speech recognition, gesture instead of using the mouse.
\item Taking over tasks from the user and providing help on using new and complex systems.
\end{itemize}
In this project we will explore the landscape of the techniques used currently to solve the above problems and implement a variation of it where applicable.

\subsubsection{Applications using Adaptive Concepts}
Adaptive System Concepts has been extensively reviewed by various researchers, Benyon and Murray\cite{benyon1993applying}, Norcio and Stanley\cite{norcio1989adaptive} have all provided useful reviews. One of the issues with areas such as Adaptive Systems is that identical concepts have come out of different disciplines. These disciplines adopt their own terminology which makes the comparison and generalisation problematic. The list of systems provided below depict some of the various systems that can be described as ``intelligent''.
\subsubsection{Intelligent User Interfaces}
As mentioned earlier, IUI is a subfield of Human-Computer Interaction; Adaptive User Interfaces(AUI) is a subtype of IUI. 
\subsection{Applications using Adaptive Concepts}
Adaptive System Concepts has been extensively reviewed by various researchers, Benyon and Murray\cite{benyon1993applying}, Norcio and Stanley\cite{norcio1989adaptive} have all provided useful reviews. One of the issues with areas such as Adaptive Systems is that identical concepts have come out of different disciplines. These disciplines adopt their own terminology which makes the comparison and generalisation problematic. The list of systems provided below depict some of the various systems that can be described as ``intelligent''.
\subsubsection{Intelligent User Interfaces}
As mentioned earlier, IUI is a subfield of Human-Computer Interaction; Adaptive User Interfaces(AUI) is a subtype of IUI.A ``normal'' interface simply defines a channel of communication between a human user and a machine, whereas an ``intelligent'' one adapts to the user, communicates with the user and solves problems for the user.
However, not all intelligent interfaces have the ability to learn and solve problems. Many interfaces we call intelligent focuses on the communication channels between the user and machine \cite{patrick2003intelligent}.
\subsubsection{Natural Language Systems}
Natural Language systems as the name implies, try to adapt to the user by generating text appropriate to the specific query and characteristics of individual users, much like Apple Inc's Siri \cite{website:SIRI}. These systems approach this problem by inferring the user's needs and focus of attention from the use of natural language \cite{benyon1993adaptive}
\subsubsection{Intelligent Tutoring Systems}
ITS systems of the notion that given a student(s) and topic(s) a computer system can alleviate the variance of human-based teaching skills and can determine the best manner in which to present individually targeted instruction in a constrained subject domain \cite{benyon1993adaptive}. ITS are analogous to AES ( Adaptive Educational Systems); AES monitor the important learner characteristics and makes the appropriate adjustments to support and enhance learning experience for the learner \cite{shute2012adaptive}.
\subsection{Adaptivity at a cost}
Complete AUI are hard to come by, this can be attributed to the fact that, the unpredictability and autonomy required for a complete aui reduces a systems usability 
another contributing factor is that users suffer difficulty in forming ade- quate mental models of such systems. there has been proposals on creating support systems to aide users form adequate mental models \cite{paymans2004usability}. A ``normal'' interface simply defines a channel of communication between a human user and a machine, whereas an ``intelligent'' one adapts to the user, communicates with the user and solves problems for the user.
However, not all intelligent interfaces have the ability to learn and solve problems. Many interfaces we call intelligent focuses on the communication channels between the user and machine \cite{patrick2003intelligent}.
\subsubsection{Natural Language Systems}
Natural Language systems as the name implies, try to adapt to the user by generating text appropriate to the specific query and characteristics of individual users, much like Apple Inc's Siri \cite{website:SIRI}. These systems approach this problem by inferring the user's needs and focus of attention from the use of natural language \cite{benyon1993adaptive}
\subsubsection{Intelligent Tutoring Systems}
ITS systems of the notion that given a student(s) and topic(s) a computer system can alleviate the variance of human-based teaching skills and can determine the best manner in which to present individually targeted instruction in a constrained subject domain \cite{benyon1993adaptive}. ITS are analogous to AES ( Adaptive Educational Systems); AES monitor the important learner characteristics and makes the appropriate adjustments to support and enhance learning experience for the learner \cite{shute2012adaptive}.
\subsubsection{Adaptivity at a cost}
Complete AUI are hard to come by, this can be attributed to the fact that, the unpredictability and autonomy required for a complete AUI reduces a systems usability another contributing factor is that users suffer difficulty in forming ade- quate mental models of such systems. there has been proposals on creating support systems to aide users form adequate mental models \cite{paymans2004usability}.
\subsection{Summary}
