\section{Conclusion}
\subsection{Introduction}
 In this paper we seen the importance of collaborative work with regards to the classroom and the impact it has on the pupils or students. We have also looked at the theories and research conducted on the seating arrangement in the classroom and how a pupil's performance in a classroom is a contributing factor in their overall academic performance and the need for tools to help teachers with this task.

We also discussed existing systems that aim to help teachers with the task of seating arrangement and where they fall short in achieving their aim. The paper also talks about adaptive user interfaces, the techniques used and the need for systems with adaptive user interfaces with regards to a tool that aims to help teachers with their seating arrangement tasks. Based on these theories and techniques we have designed and implemented a classroom management tool that takes advantage of the shortcomings of existing systems to create a system that improves on its interaction with the user based on \emph{Adaptive User Interface} techniques and adheres to proven classroom theories.

We conclude the paper and project in this chapter by giving an overview, the contributions and further work that can improve the system developed in this project.
\subsection{Project Overview}
 In the first(1) chapter of the paper we looked the problem the project aims to solve, the solution, the main concepts behind the solution and how we perceived the system developed as a result of the project to work.
 
 The second(2) chapter presented a detailed background on the subject of collaboration and classroom management as well as \emph{Adaptive User Interface} techniques. It also looked at existing systems that aim to solve the same problem as this project.
 
 In the third(3) chapter the paper discussed the requirements of this project. It talks about what the system should and must do by presenting the concept of user requirements, functional requirements and non-functional requirements.
 
 The fourth(4) chapter gives a detailed look at the design of the system based on the requirements presented in the chapter three (3). 
 
 Chapter five(5) presented the system architecture; how the design presented in chapter four(4) is put together to develop the system in this project.
 
 Chapter six(6) looked at how the design and architecture are implemented in this project. It then discusses some of the algorithms behind the core components of the system developed in this project.
 
 Chapter seven(7) discussed how well the system developed in this project performed under various tests. It also talks about the methods used in testing and the results of the tests.
\subsection{Contributions}
\subsubsection{Teaching Profession}
In the literature review chapter of the paper we looked at existing systems that aim to help teachers in their seating arrangement tasks and where they fail to achieve this aim. These systems do not only fail but also make it difficult for teachers to gain access; they are either sold on subscription bases at a really high price and focus on selling to schools instead of to individual teachers.

This project's contribution to the teaching profession is a system that does not only provide a clean interface but one whose primary target is the teacher as classroom management is under the control of the teacher.  

\subsubsection{Pupil Performance}
The aim of this project is to drive and encourage collaboration and improve pupil performance as a result giving teachers the tool to manage their classroom.Addressing how pupil are arranged in the classroom would improve pupil performance,a major contribution of this project to the future.
\subsubsection{UM Algorithm}
The \emph{User Model} algorithms used in this project is among the contributions made by this project. This project produced a way of modeling users on a web application that can be adopted by the masses as a refined open source technology.
 
\subsection{Further Work}
This sections presents future work that can be done to improve on this project.
\subsubsection{Desirable User Requirements}
In chapter \ref{sec:designRequirements} we presented the requirements of the project. Some of these requirements were marked as desirable or of low priority, these requirements were not met. The addition of these requirements could improve the overall user experience. The requirements are as listed below:
\begin{itemize}
    \item Delete year groups - The ability to delete a year group from menu tab
    \item Edit year group - The ability to edit a year group name after adding it to the system.
    \item Edit class group name - The ability to edit a class' name after adding it to the system.
\end{itemize}

The structure is in place to enable these functionalities, a pop-up modal similar to the modals used to rate the seating plans could be bound to the class and year group elements' click event. This modal would then submit its content or the action selected to firebase.

\subsubsection{Google Spreadsheet Integration}
The primary way of the user to add pupil data to the system is to upload CSV files they already have in their local system. This could be improved by integrating one of Google's API for spreadsheet to provide an automated process of upload data into the system we developed in this project.
\subsubsection{Report Generation}
Generating graphical reports in the form of graphs and charts to illustrate the user's seating arrangement history and optimal arrangements they use would also improve the system we have developed. 

A way to approach this could involve another canvas context; much like the classroom canvas and render the charts and or graphs based on the information from the history stack.
\subsubsection{Extend UM Algorithm}
Currently the system we have developed relies on a simple rule based algorithm and a light weight monitoring process to create a user model for the user. 

This could be extended with Machine Learning(ML) techniques such as K-Nearest Neighbour (KNN); this supervised learning technique which given a data set to learn from can identify the type of any new anonymous data introduced into the set. 

With the concept of this algorithm in mind the \emph{extend UM model}, can be trained with various seating arrangements frequently used by teachers. From this it can place any seating arrangement into a category, instead of iterating the seating arrangements to discover patterns based on probability.  
 
\subsection{Concluding Remarks} 
This project has looked at the importance of collaboration in the classroom, the impact it has on pupil's performance and the need to provide teachers with the necessarily tools in order to encourage efficient and effective seating arrangements in the classroom. It also addressed the issue with existing systems, where the systems are rigid and does not adapt to the user's needs and preferences, by designing and implementing a system that adopts \emph{adaptive user interface} techniques to improve on its interaction with the user.

The evaluation of the system, however inconclusive, has shown that adaptive interfaces techniques can help reduce the level of cognitive skills employed by teachers when using existing systems. 