\section{Introduction}
\subsection{Project Aims, Objectives and Introduction} 
The notion of user interfaces or software applications adapting to its users has been around for several years, gathering an enormous interest in the research community. Over the years a lot of technological advancements have been made in addtion to standards and tecniques to achieve adaptiveness on the user facing or interface layer of software applications.

In recent years,the need for applications to proactively implement such techniques has increased, especially in the Enterprise Software industry where developers strive to develop softwares that are highly efficient and almost completely eradicates the potential of human errors and the costs that come with it. By so doing, users of the software application must all use it the same way, information is presented the same way and must all reason about the information being present in more or less the same manner in order to get any useful result out of using the said application.

These applications disregard the fact that people or the users will form different mental models of their interactions with the system and have different ways of learning and retaining knowledge and skills. For example if the user is left or right brained.[The split brain in man].

The teaching profession is one of the oldest professions and as such belongs well documented and structured industry; teachers perform very specific but arduous daily tasks in addition to teaching. There have been numerous applications developed over the years to aide their work but these applications fall under \emph{Enterprise Software} where the need to produce cost effective but applications that are extremely efficient at achieving their functional superceeds the needs of the individual user.

Teachers like many other users that happen to use such applications spend hours if not weeks learning how to use these applications, instead of spending those hours on teaching materials; these applications more often than not are very rigid and do not alter their layout with the exception of perhaps the content to suit the current user.

In this project we develop a user interface system for a classroom management application that changes its layout structure and elements to suit the needs of the user (teacher) and also provides the flexibility for the user to change the interface structure and elements themselves.

We prove that this can be achieved by using Web components[cite web components] structure and implementing these elements as reusable components that subscribe to Adaptive User Interface paradigms and techniques[Refer to techniques].

We have built a system comprising several of these components, task and user models as well as intelligent agents to achieve a level of adaptiveness. The result is a system that puts the user first by altering its layout to suit the current user.

In the next section we will talk about the background Adaptive User Interface and review some relevant literature to this project, discussing the various and current trends in Adaptive User Interfaces, existing projects that aim to solve this problem and explore the advantages and disadvantages of . This will be followed by the discussion of the methodologies and the approaches used in this project and then we conclude by looking at what this project has achieved with regards to what it initially set out to do and what can be done to improve on it. 
\subsection{Background and Literature Survey} \label{sub:background}
 It gives an overall picture about the work with a clear review of the relevant literature.  The background of the project should be given.  What have been done to deal with the problem should be stated clearly.  The pros and cons of various existing algorithms and approaches should be stated as well.  Differences between your proposed method and the existing ones should be briefly described.

The following links may help on the literature review:
IEEE Xplore digital library: a resource for accessing IEEE published scientific and technical publications (You must be with King's network to get access to the digital library)
ScienceDirect.com: an electronic database offering journal papers not published by IEEE (You must be with King's network to get access to the database)
