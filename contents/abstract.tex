\section*{Abstract}

Working in pairs and groups has been found to be a key factor underlying student performance and experience. As such, teachers need tools that will enable them to group their pupils into the most efficient and effective pair and group arrangements (which, in practice, is translated as a classroom's seating plan) and monitor outcomes. These tools should be flexible and adaptive in order to suit different users.


The project reported in this thesis describes the design, implementation and evaluation of a classroom management system. The system applies existing theories of effective seating arrangements to create rules that are used in the recommendation and evaluation of seating arrangements. In addition, the system monitors user activity and behaviour in order to update its rules and recommendations, as well as its user interface components.
